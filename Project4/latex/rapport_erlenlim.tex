\documentclass[aps,reprint]{revtex4-1}
% Engine-specific settings
% Detect pdftex/xetex/luatex, and load appropriate font packages.
% This is inspired by the approach in the iftex package.
% pdftex:
\ifx\pdfmatch\undefined
\else
    \usepackage[T1]{fontenc}
    \usepackage[utf8]{inputenc}
\fi
% xetex:
\ifx\XeTeXinterchartoks\undefined
\else
    \usepackage{fontspec}
    \defaultfontfeatures{Ligatures=TeX}
\fi
% luatex:
\ifx\directlua\undefined
\else
    \usepackage{fontspec}
\fi
% End engine-specific settings
\usepackage[english]{babel}
\usepackage{csquotes}
% \usepackage[backend=biber, sortcites]{biblatex}
\usepackage{url}
\usepackage{textcomp}
\usepackage[usenames,dvipsnames,svgnames, table]{xcolor}
\usepackage[font={scriptsize}]{caption}
\usepackage{amsmath} \usepackage{amsthm} \usepackage{amsfonts}
\usepackage{amssymb}
\usepackage{enumerate}
\usepackage{tikz} \usepackage{float}
\usepackage[procnames]{listings}
\usepackage{pstool} \usepackage{pgfplots}
\usepackage{wrapfig} \usepackage{graphicx} \usepackage{epstopdf}
\usepackage{afterpage}
\usepackage{physics}
\usepackage{multirow}
\usepackage{gensymb}
\usepackage{algorithm}
\usepackage{microtype}
\usepackage[noend]{algpseudocode}
\usepackage{xcolor,colortbl}
\usepackage{microtype}
\usepackage{geometry}
\usepackage{hyperref}
\usepackage{graphicx}
\usepackage{caption}
\usepackage{subcaption}
\usepackage{lipsum}
% \usepackage{pythontex}
% \usepackage{authblk}
\usepackage{nth}
\usepackage{siunitx}
% \usepackage[toc,page]{appendix}
\floatstyle{plaintop}
\restylefloat{table}

% Custom commands
\newcommand{\unit}[1]{\:\mathrm{#1}}
\newcommand{\noref}[1]{\hyperref[#1]{\ref*{#1}}}
\newcommand{\nonref}[1]{\hyperref[]{\ref*{#1}}}
\newcommand\blankpage{%
  \null
  \thispagestyle{empty}%
  \addtocounter{page}{-1}%
  \newpage}

% Default fixed font does not support bold face
\DeclareFixedFont{\ttb}{T1}{txtt}{bx}{n}{7} % for bold
\DeclareFixedFont{\ttm}{T1}{txtt}{m}{n}{7}  % for normal

\newcommand\numberthis{\addtocounter{equation}{1}\tag{\theequation}}
\DeclareCaptionFont{white}{\color{white}}
\DeclareCaptionFormat{listing}{\colorbox{gray}{\parbox{\columnwidth}{#1#2#3}}}
\pgfplotsset{compat=1.14} %TODO: Setting this removed several error messages, should it be here!?


% Biber for references
% \bibliographystyle{aipauth4-1}

\begin{document}
\sisetup{detect-all}
\title{Ising (Or should I say Ezing?)}
\author{Erlend Lima}
\thanks{All code related to or referred by this paper was written in
  collaboration with Frederik J. Mellbye. This paper itself is written by
 solely the author.}
\affiliation{University of Oslo, Oslo, Norway \\ Source code available at: \url{https://github.com/Caronthir/FYS3150/tree/master/Project4}}
\date{\today}

\begin{abstract}
Once again the abstract abstractifies the abstractness of the abstract mathematical
equations than govern the universe.
\end{abstract}
\maketitle
\tableofcontents
\makeatletter
\let\toc@pre\relax
\let\toc@post\relax
\makeatother

\newpage

\section{Introduction}
\label{sec:introduction}

\section{Theory}
\label{sec:theory}

\subsection{The Ising Model}
\label{sec:isingmodel}

In his 1924 PhD thesis, Ernt Ising introduced 
solved the one dimensional Ising model devised by his adviser, Wilhelm Lenz. The
model describes a lattice of discrete spins taking binary values. modeling real
world phenomena such as ferromagnetism and phase transitions. Analytical
solutions exists for one and two dimensional Ising models, but it also lends
itself to numerical solutions through Monte Carlo simulations. 



\section{Method}
\label{sec:method}

\section{Results}
\label{sec:results}

\section{Discussion}
\label{sec:discussion}

\section{Conclusion}
\label{sec:conclusion}

\bibliography{references}
\blankpage
\appendix
\section{Appending appendices is easy with the appendix package.}
\blankpage
\end{document}

% Local Variables:
% TeX-engine: luatex
% End:
