\documentclass[aps,reprint]{revtex4-1}
% Engine-specific settings
% Detect pdftex/xetex/luatex, and load appropriate font packages.
% This is inspired by the approach in the iftex package.
% pdftex:
\ifx\pdfmatch\undefined
\else
    \usepackage[T1]{fontenc}
    \usepackage[utf8]{inputenc}
\fi
% xetex:
\ifx\XeTeXinterchartoks\undefined
\else
    \usepackage{fontspec}
    \defaultfontfeatures{Ligatures=TeX}
\fi
% luatex:
\ifx\directlua\undefined
\else
    \usepackage{fontspec}
\fi
% End engine-specific settings
\usepackage[english]{babel}
\usepackage{csquotes}
% \usepackage[backend=biber, sortcites]{biblatex}
\usepackage{url}
\usepackage{textcomp}
\usepackage[usenames,dvipsnames,svgnames, table]{xcolor}
\usepackage[font={scriptsize}]{caption}
\usepackage{amsmath} \usepackage{amsthm} \usepackage{amsfonts}
\usepackage{amssymb}
\usepackage{enumerate}
\usepackage{tikz} \usepackage{float}
\usepackage[procnames]{listings}
\usepackage{pstool} \usepackage{pgfplots}
\usepackage{wrapfig} \usepackage{graphicx} \usepackage{epstopdf}
\usepackage{afterpage}
\usepackage{physics}
\usepackage{multirow}
\usepackage{gensymb}
\usepackage{algorithm}
\usepackage{microtype}
\usepackage[noend]{algpseudocode}
\usepackage{xcolor,colortbl}
\usepackage{microtype}
\usepackage{geometry}
\usepackage{hyperref}
\usepackage{graphicx}
\usepackage{caption}
\usepackage{subcaption}
\usepackage{lipsum}
% \usepackage{pythontex}
% \usepackage{authblk}
\usepackage{nth}
\usepackage{siunitx}
% \usepackage[toc,page]{appendix}
\floatstyle{plaintop}
\restylefloat{table}

% Custom commands
\newcommand{\unit}[1]{\:\mathrm{#1}}
\newcommand{\noref}[1]{\hyperref[#1]{\ref*{#1}}}
\newcommand{\nonref}[1]{\hyperref[]{\ref*{#1}}}
\newcommand\blankpage{%
  \null
  \thispagestyle{empty}%
  \addtocounter{page}{-1}%
  \newpage}

% Default fixed font does not support bold face
\DeclareFixedFont{\ttb}{T1}{txtt}{bx}{n}{7} % for bold
\DeclareFixedFont{\ttm}{T1}{txtt}{m}{n}{7}  % for normal

\newcommand\numberthis{\addtocounter{equation}{1}\tag{\theequation}}
\DeclareCaptionFont{white}{\color{white}}
\DeclareCaptionFormat{listing}{\colorbox{gray}{\parbox{\columnwidth}{#1#2#3}}}
\pgfplotsset{compat=1.14} %TODO: Setting this removed several error messages, should it be here!?


% Biber for references
% \bibliographystyle{aipauth4-1}

\begin{document}
\sisetup{detect-all}
\title{Simulation of the solar system with numerical methods}
\author{Erlend Lima}
\author{Frederik J. Mellbye}
\affiliation{University of Oslo, Oslo, Norway \\ Source code available at: \url{https://github.com/Caronthir/FYS3150/tree/master/Project3}}
\date{\today}

\begin{abstract}
Abstract abstract.
\end{abstract}
\maketitle
\tableofcontents
\makeatletter
\let\toc@pre\relax
\let\toc@post\relax
\makeatother

\newpage

\section{Introduction}
\label{sec:introduction}
Intro dintro mintro pintroduction?

For the simple case of two celestial bodies it is trivial to solve the time
development of the system anlytically. When more objects are added however, the equations
have no analytical solution, and numerical methods are required to find
approximated solutions. This is popularly called a n-body problem. In this project, the classic forward Euler and
Velocity Verlet algorithms are used to simulate the solar system, and their
accuracies and runtimes are compared.
\section{Theory}
\label{sec:theory}
For planetary motion, the only force present is gravity. Newton's law of gravitation
states that any two objects in the universe attract each other by a force given
by
\begin{align}
  F_G = \frac{G m_1 m_2}{r^2}
\end{align}
where $m_1$ and $m_2$ are the masses of the objects, $G$ is the
gravitational constant and $r$ is the distance between the objects. In the
case examined in this paper, it is assumed that the solar mass is much larger
than the planet masses ($M_\odot \gg M_\text{planet}$), and the solar
movement is therefore ignored. Then, for a planet orbiting the sun, the
planet position as a function of time is described by Newton's second law:
\begin{align*}
  \dv[2]{\mathbf{x}}{t} = \frac{\mathbf{F}_G}{M}
\end{align*}
where $M$ is the planet mass. This is equivalent to one ODE (ordinary differential
equation) for each direction component, i.e. three separate equations in the
three-dimensional case.

For the numerical methods the above second order ODE is more convenient to
work with when written as a set of two coupled first order equations. These
are given by
\begin{align}
  \begin{split}
  \label{eq:coupledequations}
  \dv{x}{t} &= v(x,t) \\
  \dv{v}{t} &= \frac{F(x,t)}{M} = a(x,t)
  \end{split}
\end{align}
The algorithms used are both based on Taylor expansions, the forward Euler
scheme approximates the derivatives to first order, while the Verlet
method combines a second order approximation with the fact that the acceleration
is only position dependent.
\subsection{Forward Euler}
The Euler method (forward Euler) is the most basic numerical method to solve
ODEs. This widely known algorithm applied to ~\ref{eq:coupledequations} is given by
\begin{align*}
  \begin{split}
  v_{i+1} = v_{i} + a_{i}\Delta{t} \\
  x_{i+1} = x_{i} + v_{i}\Delta{t}
\end{split}
\end{align*}
These equations are derived by simply taylor expanding the derivatives and omitting
the error terms.
\subsection{Velocity Verlet}
The velocity Verlet method is based on Taylor-expanding the position and velocity
to second order, and specialized for cases where the acceleration is only
dependent on position. The algorithm (see ~\ref{sec:velocityverlet} for details)
applied to the problem examined in this paper is given by the following equations:
\begin{align}
  \begin{split}
    x_{i+1} &= x_i + v_i + \frac{\Delta{t}^2}{2} a_i \\
    v_{i+1} &= v_i + \frac{\Delta{t}}{2}(a_{i+1} + a_{i})
  \end{split}
\end{align}
Note that this algorithm assumes $a_{i+1}$ is only dependent on position $x_{i+1}$
and not velocity $v_{i+1}$. This is because the acceleration in the next step
is required to calculate the velocity in the next step. Because $a_{i+1} = a(x_{i+1})$,
the position in the next step is computed prior to the velocity. This algorithm
therefore seems suited to solve for planetary motion, because gravity is only
position dependent and the only force that acts within the system.

\subsection{Initial conditions}

\subsection{Errors}
\section{Method}
\label{sec:method}
\subsection{Choice of units}
\section{Results}
\label{sec:results}

\section{Discussion}
\label{sec:discussion}

\section{Conclusion}
\label{sec:conclusion}

\bibliography{references}
\blankpage
\appendix
\section{Derivation of velocity Verlet algorithm}
\label{sec:velocityverlet}
The Velocity Verlet method is derived for one dimension ($x$), but is easily
generalized for three dimensions by simply repeating the procedure with $y$
or $z$ instead of $x$. The derivation can also be generalized to three dimensions
simply by considering $x$ as a vector $\mathbf{x} = (x, y, z)$.
The position and time is discretized to $n$ integration points so that
\begin{align*}
  x(t) &\rightarrow x(t_i) &&= x_i \\
  t &\rightarrow t_i &&= t_0 + i\Delta{t}
\end{align*}
where $\Delta{t} = \frac{t_\text{final} - t_0}{n-1}$ the time step size. The velocity
is similarly discretized to each time point so that $v(t) \rightarrow v_i$.
Taylor expanding the position and velocity yields
\begin{align*}
  x_{i+1} &= x_i + \Delta{t} x'_i + \frac{\Delta{t}^2}{2} x''_i + O(\Delta{t}^3) \\
  v_{i+1} &= v_i + \Delta{t} v'_i + \frac{\Delta{t}^2}{2} v''_i + O(\Delta{t}^3)
\end{align*}
The second derivative of the velocity is approximated using Euler's method, so
\begin{align*}
  \Delta{t} v''_i = v'_{i+1} - v'_{i} = a_{i+1} - a_{i}
\end{align*}
where $a_i$ is the acceleration at $t = t_i$. With this approximation for the
velocity second derivative, the Taylor expansions are given by
\begin{align*}
  v_{i+1} &= v_i + \Delta{t} a_i + \frac{\Delta{t}}{2} (a_{i+1} - a_i) + O(\Delta{t}^3) \\
          &= v_i + \frac{\Delta{t}}{2} (a_{i+1} + a_{i}) + O(\Delta{t}^3)
\end{align*}
For the numerical algorithm the error terms are omitted and the resulting
algorithm is
\begin{align}
  \begin{split}
    x_{i+1} &= x_i + v_i + \frac{\Delta{t}^2}{2} a_i \\
    v_{i+1} &= v_i + \frac{\Delta{t}}{2}(a_{i+1} + a_{i})
  \end{split}
\end{align}
\blankpage
\end{document}

% Local Variables:
% TeX-engine: luatex
% End:
