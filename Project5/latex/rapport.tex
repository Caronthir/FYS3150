\documentclass[aps,reprint]{revtex4-1}
% Engine-specific settings
% Detect pdftex/xetex/luatex, and load appropriate font packages.
% This is inspired by the approach in the iftex package.
% pdftex:
\ifx\pdfmatch\undefined
\else
    \usepackage[T1]{fontenc}
    \usepackage[utf8]{inputenc}
\fi
% xetex:
\ifx\XeTeXinterchartoks\undefined
\else
    \usepackage{fontspec}
    \defaultfontfeatures{Ligatures=TeX}
\fi
% luatex:
\ifx\directlua\undefined
\else
    \usepackage{fontspec}
\fi
% End engine-specific settings
\usepackage[english]{babel}
\usepackage{csquotes}
% \usepackage[backend=biber, sortcites]{biblatex}
\usepackage{url}
\usepackage{textcomp}
\usepackage[usenames,dvipsnames,svgnames, table]{xcolor}
\usepackage[font={scriptsize}]{caption}
\usepackage{amsmath} \usepackage{amsthm} \usepackage{amsfonts}
\usepackage{amssymb}
\usepackage{enumerate}
\usepackage{tikz} \usepackage{float}
\usetikzlibrary{patterns}
\usepackage[procnames]{listings}
\usepackage{pstool} \usepackage{pgfplots}
\usepackage{wrapfig} \usepackage{graphicx} \usepackage{epstopdf}
\usepackage{afterpage}
\usepackage{physics}
\usepackage{multirow}
\usepackage{gensymb}
\usepackage{algorithm}
\usepackage{microtype}
\usepackage[noend]{algpseudocode}
\usepackage{xcolor,colortbl}
\usepackage{microtype}
\usepackage{geometry}
\usepackage{hyperref}
\usepackage{graphicx}
\usepackage{caption}
\usepackage{subcaption}
\usepackage{lipsum}
% \usepackage{pythontex}
% \usepackage{authblk}
\usepackage{nth}
\usepackage{siunitx}
% \usepackage[toc,page]{appendix}
\floatstyle{plaintop}
\restylefloat{table}

% Custom commands
\newcommand{\unit}[1]{\:\mathrm{#1}}
\newcommand{\noref}[1]{\hyperref[#1]{\ref*{#1}}}
\newcommand{\nonref}[1]{\hyperref[]{\ref*{#1}}}
\newcommand\blankpage{%
  \null
  \thispagestyle{empty}%
  \addtocounter{page}{-1}%
  \newpage}

% Default fixed font does not support bold face
\DeclareFixedFont{\ttb}{T1}{txtt}{bx}{n}{7} % for bold
\DeclareFixedFont{\ttm}{T1}{txtt}{m}{n}{7}  % for normal

\newcommand\numberthis{\addtocounter{equation}{1}\tag{\theequation}}
\DeclareCaptionFont{white}{\color{white}}
\DeclareCaptionFormat{listing}{\colorbox{gray}{\parbox{\columnwidth}{#1#2#3}}}
\pgfplotsset{compat=1.14} %TODO: Setting this removed several error messages, should it be here!?
\def\axislength{.5}

% Colors
\colorlet{MyColorOne}{blue!35}
\newcommand{\lightercolor}[3]{% Reference Color, Percentage, New Color Name
    \colorlet{#3}{#1!#2!white}
}
\newcommand{\darkercolor}[3]{% Reference Color, Percentage, New Color Name
    \colorlet{#3}{#1!#2!black}
}
\lightercolor{MyColorOne}{50}{MyColorOneLight}
\darkercolor{MyColorOne}{50}{MyColorOneDark}

\newcommand{\bigO}[1]{\ensuremath{\mathop{}\mathopen{}\mathcal{O}\mathopen{}\left(#1\right)}}

% Biber for references
% \bibliographystyle{aipauth4-1}

\begin{document}
\sisetup{detect-all}
\title{Numerical solutions to the diffusion equation in one and two dimensions}
\author{Erlend Lima}
\author{Frederik J. Mellbye}
\affiliation{University of Oslo, Oslo, Norway \\ Source code available at: \url{https://github.com/Caronthir/FYS3150/tree/master/Project3}}
\date{\today}

\begin{abstract}
Abstract.
\end{abstract}
\maketitle
\tableofcontents
\makeatletter
\let\toc@pre\relax
\let\toc@post\relax
\makeatother

\newpage

\section{Introduction} \label{sec:introduction}
The diffusion equation is a partial differential equation that appears frequently in
multiple fields of science. Among others, heat transfer processes, Markov processes
and applications in life science are examples of situations where the equation describes
some phenomena. Furthermore, the numerical methods that are used to solve the diffusion
equation are easily modified to solve similar equations, which demonstrates the relevance
of investigating the methods and attributed stability conditions and errors.

In this paper the diffusion equation is derived for a one-dimensional shear-stress
driven Couette flow from the incompressible Navier-Stokes equations. The resulting
equation along with the boundary and initial conditions are scaled and discretized.
Three numerical methods are employed to solve the equation, namely the explicit Forward and
implicit Backward Euler methods, and a combination of the two, the Crank-Nicolson scheme.
The solutions from the numerical methods are compared to the analytical solution, which
shows a close agreement. The solver is then extended to solve the two dimensional
diffusion equation, which for instance can be used to describe heat flow in a
plate with edges held at given temperatures.
\section{Theory} \label{sec:theory}
\subsection{Deriving the diffusion equation for a Couette flow}
A Couette flow is a drag-induced flow between either parallel plates or concentric
rotating cylinders and therefore differs from pressure driven flows such as the
Hagen-Poiseuille flow. It is assumed that the flow is laminar and of an incompressible
Newtonian fluid of density $\rho$ and vicosity $\mu$ between infinite parallel
plates located at $y = 0$ and $y = L$. The situation is illustrated in figure~\ref{fig:couette}.
Initially, the plates and the fluid is at rest, but at some time $t = t_0$
the upper plate is given a horizontal velocity $U$.
\begin{figure}[H]
  \centering
  \begin{tikzpicture}
    % \draw[help lines, thick, dashed, opacity=0.5] (0,0) grid (8,4);
    \draw (0,0) -- (6,0) node[anchor=north] {$y = 0$};
    \draw (0,3) -- (6,3) node[anchor=south] {$y = L$};

    \draw[->, thick] (2,3.2) -- (4,3.2);
    \node at (3,3.4) {$U$};
    \node at (5,1.5) {$\rho, \mu$};

    \draw[MyColorOne,dashed] (4,3) -- (2,0);
    \draw[MyColorOne,dashed] (2,0) -- (2,3);
    \foreach \y in {0.5,1,...,2.5} {%
      \draw[->, MyColorOneDark, thick] (2,\y) -- (2 + 2*\y/3,\y);
    }%

    \coordinate (O) at (0,0.1);
    \coordinate (x) at (\axislength,0.1);
    \coordinate (y) at (0,\axislength+.1);
    \draw[->] (O) -- (x) node[above] {$x$};
    \draw[->] (O) -- (y) node[right] {$y$};

    \fill[pattern=north west lines] (0,-.1) rectangle ++(6,.1);
    \fill[pattern=north west lines] (0,3) rectangle ++(6,.1);
  \end{tikzpicture}
  \caption{Couette flow between two parallel plates. The horizontal distance
  scale is $\gg L$, so the plates are considered infinite in the $x$ and $z$
  directions. The flow is here at a fully developed stage.}
  \label{fig:couette}
\end{figure}
The Navier-Stokes equations govern the time development of the fluid velocity field.
The momentum equation reads
\begin{align*}
  \pdv{\vb{u}}{t} + (\vb{u} \cdot \nabla) \vb{u} = - \frac{1}{\rho} \nabla p + \nu \nabla^2 \vb{u}
\end{align*}
where $\nu = \mu/\rho$ is the kinematic viscosity, and the incompressibility
condition is
\begin{align*}
  \nabla \cdot \vb{u} = 0
\end{align*}
The only force acting on the system is the tangential shear stress in the
horizontal direction, and the velocity field is therefore assumed to be on the
form $\vb{u} = (u,v = 0,w = 0)$. The pressure gradient term is eliminated because the
pressure is constant, and the inertial term vanishes because of the incompressibility
condition. The equations reduce to
\begin{align}
  \pdv{u}{t} = \nu \pdv[2]{u}{y}
\end{align}
which is the diffusion equation. The no-slip boundary condition applies to both
surfaces, so
\begin{align*}
  u(y = L) &= U \\
  u(y = 0) &= 0
\end{align*}
To generalize the results the equations are scaled and written in terms of
a spatial variable $x$ instead of $y$:
\begin{align}
  \pdv{\hat{u}}{\hat{t}} = \pdv[2]{\hat{u}}{\hat{x}}
\end{align}
where $\hat{x} = x / L$ and $\hat{u} = u / U$ and thus $\hat{t} = t / (L^2 / \nu)$ are dimensionless quantities.
The initial conditions are converted to $\hat{u}(\hat{x},0) = 0$ for $0 < \hat{x} < 1$,
and $\hat{u}(0,\hat{t}) = 0$ and $\hat{u}(1,\hat{t}) = 1$ at the boundaries for $\hat{t} > 0$.
Only the dimensionless equation is considered hereafter, with the hats omitted.
\subsection{Developing the numerical algorithms}
In this project three numerical methods for solving partial differential equations (PDEs)
are used. The methods are the explicit forward and implicit backward Euler
algorithms, along with the implicit Crank-Nicolson scheme. For all three schemes
position, time and thus velocity is discretized:
\begin{align*}
  t &\rightarrow t_j = t_0 + j \Delta{t} \\
  x &\rightarrow x_i = x_0 + i \Delta{x} \\
  u(x,t) &\rightarrow u(x_i, t_i)
\end{align*}
where $0 < i < n + 1$ and $j \geq 0$. The position step size is given by $\Delta{x} = 1 /(n+1)$.
The notation $u(x_i,t_i) = u_{i,j}$ is used. The following derivations are based on
section 10.2 in \cite{mortenjensen}, which has more detailed explanations.
\subsubsection{Explicit forward Euler scheme}
Perhaps the simplest method for forwarding the positions as a function of time
is the forward Euler method. The derivatives are approximated using
\begin{align*}
  \pdv{u}{t} &= \frac{u_i,j+1 - u_i,j}{\Delta{t}} + \bigO{\Delta{t}} \\
  \pdv[2]{u}{x} &= \frac{u_{i+1,j} - 2 u_{i,j} + u_{i-1, j}}{\Delta{x}^2} + \bigO{\Delta{x}^2}
\end{align*}
Inserting these expressions in the diffusion equation, omitting the error terms
and setting $\alpha = \Delta{t} / \Delta{x}^2$ yields the explicit scheme
\begin{align} \label{eq:explicitscheme}
  u_{i,j+1} = \alpha u_{i_1,j} + (1-2\alpha) u_{i,j} + \alpha u_{i+1,j}
\end{align}
Given initial and boundary values, the succeding steps can be calculated using~\ref{eq:explicitscheme},
all the terms on the right hand side are known and therefore explicitly give
the solution in the next step. The above equation can be rewritten to a matrix-vector
multiplication
\begin{align*}
  \vb{u}_{j+1} = A \vb{u}_j
\end{align*}
where the entries in $\vb{u}_j$ are the $u_i$'s at time $t_j$ and the matrix $A$ is
given by
\begin{align*}
  A = \begin{bmatrix}
        1 - 2\alpha & \alpha      & 0            & \hdots  & 0      \\
        \alpha      & 1 - 2\alpha & \alpha       & 0       & \vdots \\
        0           & \alpha      & 1 - 2\alpha  & \alpha  & \vdots \\
        \vdots      & \ddots      & \ddots       & \ddots  & \vdots \\
        0           & \hdots      & \hdots       & \alpha  & 1 - 2\alpha \\
      \end{bmatrix}
\end{align*}
The discretized PDE can therefore be written as a chain of matrix multiplications
\begin{align*}
  \vb{u}_{j} = A^{j} \vb{u}_0
\end{align*}
where $\vb{u}_0 = g(x)$ is the initial state.
\subsubsection{Implicit backward Euler}
For the implicit backward Euler method the derivation is almost identical,
the difference originates at the backward Euler time derivative expansion
\begin{align*}
  \pdv{u}{t} = \frac{u_{i,j} - u_{i,j-1}}{\Delta{t}} + \bigO{\Delta{t}}
\end{align*}
which yields the implicit steps
\begin{align*}
  u_{i,j-1} = -\alpha u_{i_1,j} + (1-2\alpha) u_{i,j} - \alpha u_{i+1,j}
\end{align*}
Note that $\alpha$ is the same as in the forward Euler scheme. As in the previous
scheme, a time step is solved through a matrix equation
\begin{align*}
  A \vb{u}_{j} = \vb{u}_{j-1}
\end{align*}
where
\begin{align*}
  A = \begin{bmatrix}
        1 + 2\alpha & -\alpha      & 0            & \hdots  & 0      \\
        -\alpha      & 1 + 2\alpha & -\alpha       & 0       & \vdots \\
        0           & -\alpha      & 1 + 2\alpha  & -\alpha  & \vdots \\
        \vdots      & \ddots      & \ddots       & \ddots  & \vdots \\
        0           & \hdots      & \hdots       & -\alpha  & 1 + 2\alpha \\
      \end{bmatrix}
\end{align*}
Each step is a backwards step, however if the matrix is constant in time it only
needs to be inverted once. The solution at time $t_j$ is therefore given by
\begin{align*}
  \vb{u}_j = A^{-j} \vb{u}_0
\end{align*}
\subsubsection{Crank-Nicolson scheme}
The Crank-Nicolson scheme combines the ideas from the above Euler methods and is
a second-order implicit method in time. Using the Taylor expansions seen in
\cite{mortenjensen}, the diffusion equation can be rewritten to
\begin{align*}
  -\alpha u_{i_1, j} &+ (2 + 2\alpha) u_{i,j} - \alpha u_{i+1,j}\\
   &= \alpha u_{i-1,j-1} + (2 - 2\alpha) u_{i, j-1} + \alpha u_{i+1,j-1}
\end{align*}
where $\alpha$ is the same as in the other methods. In matrix-vector form this
is equivalent to
\begin{align*}
  (2 I + \alpha B) \vb{u}_j = (2 I - \alpha B) \vb{u}_{j-1}
\end{align*}
where the solution vectors $\vb{u}$ are the same as before, $I$ is the identity
matrix and $B$ is given by
\begin{align*}
  B = \begin{bmatrix}
        2  & -1 & \hdots  & 0      \\
        -1 & 2  & -1 & 0       & \vdots \\
        0  & -1 & 2  & -1  & \vdots \\
        \vdots       & \ddots      & \ddots       & \ddots  & \vdots \\
        0  & \hdots  & \hdots       & -1  & 2 \\
      \end{bmatrix}
\end{align*}
Rewriting the above equation, the Crank-Nicolson scheme is
\begin{align*}
  \vb{u}_j = (2 I + \alpha B)^{-1} (2 I - \alpha B) \vb{u}_{j-1}
\end{align*}
\subsection{Algorithm stability and trunctation errors}
The algorithms have different accuracies and are limited by different stability
conditions. The solution approaches definite values if the spectral radius $\rho(A)$ of
the matrices satisfies
\begin{align} \label{eq:spectralradius}
  \rho(A) = \text{max}\{ |\lambda| : \text{det}(A - \lambda I) = 0 \} < 1
\end{align}
i.e. the maximum absolute eigenvalue can not exceed unity. From this is follows
that positive definite matrices always satisfy~\ref{eq:spectralradius}.

In addition to the stability requirement there will obviously be trunctation errors
for each scheme. The requirements are summarized in table~\ref{table:errors} and derived in the
appendix, see~\ref{sec:errors}.
\begin{table}[H]
\centering
\caption{Summarized trunctation errors and stability requirements for the schemes.}
\label{table:errors}
\begin{tabular}{p{0.31\linewidth}|p{0.34\linewidth}|p{0.28\linewidth}}
\hline
Scheme         & Trunctation Error                          & Stability requirement                     \\ \hline
Crank-Nicolson & $\bigO{\Delta{x^2}}$, $\bigO{\Delta{t^2}}$ & None                                      \\
Forward Euler  & $\bigO{\Delta{x^2}}$, $\bigO{\Delta{t}}$   & $\Delta{t} \leq \frac{1}{2} \Delta{x^2}$  \\
Backward Euler & $\bigO{\Delta{x^2}}$, $\bigO{\Delta{t}}$   & None                                      \\ \hline
\end{tabular}
\end{table}
\subsection{Analytic solution for the 1-D diffusion equation}
Analytic solutions are available for the one-dimensional diffusion equation. Assume
the differential equation is scaled and on the form
\begin{align*}
  \pdv[2]{u}{x} = \pdv{u}{t}
\end{align*}
with initial conditions
\begin{align*}
  u(x,0) = g(x) \qquad 0 < x < L
\end{align*}
and constant boundary conditions
\begin{align*}
  u(0,t) = 0 \qquad u(L,t) = 1 \qquad t \geq 0
\end{align*}
Using separation of variables, the solution is assumed to be on the form
$u(x,t) = X(x)T(t)$. Inserting this in the diffusion equation yields
\begin{align*}
  \frac{1}{X} \pdv[2]{X}{x} = \frac{1}{T} \pdv{T}{t}
\end{align*}
The left hand side of the equation is only a function of $x$, and the right hand
side similarly depends only on $t$. Thus both sides must be constant (set to $-\lambda^2$).
The PDE is therefore transformed to two ODE's:
\begin{align*}
  \pdv[2]{X}{x} + \lambda^2 X &= 0 \\
  \pdv{T}{t}    + \lambda^2 T &= 0
\end{align*}
which have general solutions
\begin{align*}
  X(x) &= A\sin{(\lambda x)} + B\cos{(\lambda x)} \\
  T(t) &= Ce^{-\lambda^2 t}
\end{align*}
The boundary conditions demand $B = 0$ and $\lambda = n \pi /L$ for $n = 1,2,\hdots$.
This yields solutions
\begin{align*}
  u(x,t) = A_n \sin{(n\pi x/L)}e^{-n^2 \pi^2 t/ L^2}
\end{align*}
Linear combinations of these solutions are also solutions because the diffusion
equation is linear. Thus
\begin{align}
  u(x,t) = \sum_{n = 1}^{\infty} A_n \sin{(n\pi x/L)}e^{-n^2 \pi^2 t/ L^2}
\end{align}
is a solution to the equation. The Fourier coefficients $A_n$ are determined
from the initial condition $u(x,0) = g(x)$. This is done by mapping the initial
configuration to the Fourier series base, through the inner product
\begin{align*}
  A_n = \frac{2}{L}\int_0^L g(x) \sin{(n \pi x / L)} \dd{x}
\end{align*}
\subsection{Two dimensions}
In two spatial dimensions, the diffusion equation is extended to
\begin{align} \label{eq:diffusion2d}
  \nabla^2 u = \pdv[2]{u}{x} + \pdv[2]{u}{y} = \pdv{u}{t}
\end{align}
where the solution $u(x,y,t)$ is a function of time $t$ and two spatial coordinates $x$ and $y$,
the problem is therefore $2 + 1$ dimensional. $\nabla^2$ is the Laplacian. The solution at
each time step is now a two-dimensional grid, which needs to be enclosed by boundary
conditions in two dimensions.
\subsubsection{Explicit scheme}
The simplest method for approximating a solution to the two-dimensional diffusion
equation is an explicit forward scheme. The derivatives are approximated as above:
\begin{align*}
  \pdv[2]{u}{x} &= \frac{u^l_{i+1,j} - 2u^l_{i,j} + 2u^l_{i-1,j}}{h^2} \\
  \pdv[2]{u}{y} &= \frac{u^l_{i,j+1} - 2u^l_{i,j} + 2u^l_{i,j-1}}{h^2} \\
  \pdv{u}{t} &= \frac{u^l_{i,j} - u^{l-1}_{i,j}}{\Delta{t}}
\end{align*}
where $i$ and $j$ are node indices in space and $l$ represents the time steps. It
is here assumed that the lattice is a square so $h = \Delta{x} = \Delta{y}$. Inserting
these expressions in~\ref{eq:diffusion2d} yields
\begin{align*}
  u_{i,j}^{l+1} = u_{i,j}^l + \alpha (u^l_{i+1,j} + u^l_{i-1,j} + u_{i,j+1}^l + u_{i,j-1}^l - 4 u^l_{i,j})
\end{align*}
where $\alpha = \Delta{t}/h^2$. The right hand side is only a function of known
quantities (current state and boundary conditions), and therefore explcitly gives
the solution in the next step.
\section{Method} \label{sec:method}
\subsection{One-dimensional implementation}
How are the matrix equations solved? Tridiag and so on.
\subsection{Two-dimensional implementation}
Same. Explicit method used.
\subsection{Error calculations}
How are the errors calculated in the code? Relative error?
\section{Results} \label{sec:results}
\section{Discussion} \label{sec:discussion}
\section{Conclusion} \label{sec:conclusion}
\bibliography{references}
\blankpage
\appendix
\section{Derivation of trunctation errors and stability conditions} \label{sec:errors}
\subsection{Explicit scheme}
The explicit scheme uses standard approximations for the derivatives which are
inserted in the diffusion equation. The first order time approximation has
an error term $\sim \Delta{t}$, and the second derivative position approximation
has an error $\Delta{x^2}$. The method is therefore second order in position and
first order in time.

For the stability condition recall equation~\ref{eq:spectralradius}, which states
that the spectral radius $\rho$ of the matrix does not exceed $1$. If the scheme is
unstable, the algorithm will amplify trunctation and round-off errors, which is
now shown to be a possibility for the explicit scheme. The matrix
is rewritten to $A = I - \alpha B$, where
\begin{align*}
  B = \begin{bmatrix}
        2  & -1 & \hdots  & 0      \\
        -1 & 2  & -1 & 0       & \vdots \\
        0  & -1 & 2  & -1  & \vdots \\
        \vdots       & \ddots      & \ddots       & \ddots  & \vdots \\
        0  & \hdots  & \hdots       & -1  & 2 \\
      \end{bmatrix}
\end{align*}
$A$ therefore has eigenvalues $\lambda_i = 1 - \alpha \mu_i$ where $\mu_i$ is
the eigenvalues of $B$. The matrix elements of $B$ are $b_{i,j} = 2\delta_{i,j} - \delta_{i+1,j} - \delta{i-1,j}$.
For component $i$ the eigenequations are
\begin{align*}
  (B\vb{x})_i = \sum_{j=1}^{n} (2\delta_{i,j} - \delta_{i+1,j} - \delta{i-1,j})x_j = 2x_i - x_{i+1} - x_{i-1} = \mu_i x_i
\end{align*}
Expanding $x$ in a basis of sines with arguments $\theta = l \pi / n + 1$, the previous equation can be rewritten as
\begin{align*}
  2(1 - \cos{(\theta)}) = \mu_i
\end{align*}
from which it is clear that the eigenvalues are $\mu_i = 2 - 2 \cos{(\theta)}$. The
spectral radius requirement is therefore
\begin{align*}
  -1 < 1 - 2 \alpha (1 - \cos{(\theta)}) < 1
\end{align*}
which is satisfied if $\alpha \leq 1/2$. Thus the stability requirement is
\begin{align}
  \frac{\Delta{t}}{\Delta{x^2}} \leq \frac{1}{2}
\end{align}
The explicit scheme for solving the one-dimensional diffusion equation is therefore
conditionally stable.

\subsection{Implicit scheme}
The implicit scheme uses the same time and position approximations as the explicit
scheme, and therefore has identical trunctation error orders. However, the matrix
is here positive definite, and the eigenvalues automatically satisfy~\ref{eq:spectralradius}.
The implicit scheme is therefore unconditionally stable.

\subsection{Crank-Nicolson scheme}
Same as implicit but second order in time.
\blankpage
\end{document}

% Local Variables:
% TeX-engine: luatex
% End:
