\documentclass[aps,reprint]{revtex4-1}
% Engine-specific settings
% Detect pdftex/xetex/luatex, and load appropriate font packages.
% This is inspired by the approach in the iftex package.
% pdftex:
\ifx\pdfmatch\undefined
\else
    \usepackage[T1]{fontenc}
    \usepackage[utf8]{inputenc}
\fi
% xetex:
\ifx\XeTeXinterchartoks\undefined
\else
    \usepackage{fontspec}
    \defaultfontfeatures{Ligatures=TeX}
\fi
% luatex:
\ifx\directlua\undefined
\else
    \usepackage{fontspec}
\fi
% End engine-specific settings
\usepackage[english]{babel}
\usepackage{csquotes}
% \usepackage[backend=biber, sortcites]{biblatex}
\usepackage{url}
\usepackage{textcomp}
\usepackage[usenames,dvipsnames,svgnames, table]{xcolor}
\usepackage[font={scriptsize}]{caption}
\usepackage{amsmath} \usepackage{amsthm} \usepackage{amsfonts}
\usepackage{amssymb}
\usepackage{enumerate}
\usepackage{tikz} \usepackage{float}
\usetikzlibrary{patterns}
\usepackage[procnames]{listings}
\usepackage{pstool} \usepackage{pgfplots}
\usepackage{wrapfig} \usepackage{graphicx} \usepackage{epstopdf}
\usepackage{afterpage}
\usepackage{physics}
\usepackage{multirow}
\usepackage{gensymb}
\usepackage{algorithm}
\usepackage{microtype}
\usepackage[noend]{algpseudocode}
\usepackage{xcolor,colortbl}
\usepackage{microtype}
\usepackage{geometry}
\usepackage{hyperref}
\usepackage{graphicx}
\usepackage{caption}
\usepackage{subcaption}
\usepackage{lipsum}
% \usepackage{pythontex}
% \usepackage{authblk}
\usepackage{nth}
\usepackage{siunitx}
% \usepackage[toc,page]{appendix}
\floatstyle{plaintop}
\restylefloat{table}

% Custom commands
\newcommand{\unit}[1]{\:\mathrm{#1}}
\newcommand{\noref}[1]{\hyperref[#1]{\ref*{#1}}}
\newcommand{\nonref}[1]{\hyperref[]{\ref*{#1}}}
\newcommand\blankpage{%
  \null
  \thispagestyle{empty}%
  \addtocounter{page}{-1}%
  \newpage}

% Default fixed font does not support bold face
\DeclareFixedFont{\ttb}{T1}{txtt}{bx}{n}{7} % for bold
\DeclareFixedFont{\ttm}{T1}{txtt}{m}{n}{7}  % for normal

\newcommand\numberthis{\addtocounter{equation}{1}\tag{\theequation}}
\DeclareCaptionFont{white}{\color{white}}
\DeclareCaptionFormat{listing}{\colorbox{gray}{\parbox{\columnwidth}{#1#2#3}}}
\pgfplotsset{compat=1.14} %TODO: Setting this removed several error messages, should it be here!?
\def\axislength{.5}


% Biber for references
% \bibliographystyle{aipauth4-1}

\begin{document}
\sisetup{detect-all}
\title{Diffusion of something}
\author{Erlend Lima}
\author{Frederik J. Mellbye}
\affiliation{University of Oslo, Oslo, Norway \\ Source code available at: \url{https://github.com/Caronthir/FYS3150/tree/master/Project3}}
\date{\today}

\begin{abstract}
Abstract.
\end{abstract}
\maketitle
\tableofcontents
\makeatletter
\let\toc@pre\relax
\let\toc@post\relax
\makeatother

\newpage

\section{Introduction} \label{sec:introduction}
Diffusion something.
\section{Theory} \label{sec:theory}
\subsection{Deriving the diffusion equation for a Couette flow}
A Couette flow is a drag-induced flow between either parallel plates or concentric
rotating cylinders and therefore differs from pressure induced flows such as the
Hagen-Poiseuille flow. It is assumed that the flow is laminar and of an incompressible
Newtonian fluid of density $\rho$ and vicosity $\mu$ between infinite parallel
plates located at $x = 0$ and $x = L$. The situation is illustrated in figure~\ref{fig:couette}.
Initially, the plates and the fluid is at rest, but at some time $t = t_0$
the upper plate is given a horizontal velocity $U$.
\begin{figure}[H]
  \centering
  \begin{tikzpicture}
    % \draw[help lines, thick, dashed, opacity=0.5] (0,0) grid (8,4);
    \draw (0,0) -- (6,0) node[anchor=north] {$y = 0$};
    \draw (0,3) -- (6,3) node[anchor=south] {$y = L$};

    \draw[->, thick] (2,3.2) -- (4,3.2);
    \node at (3,3.4) {$U$};
    \node at (5,1.5) {$\rho, \mu$};

    \draw[gray,dashed] (4,3) -- (2,0);
    \draw[gray,dashed] (2,0) -- (2,3);
    \foreach \y in {0.5,1,...,2.5} {%
      \draw[->] (2,\y) -- (2 + 2*\y/3,\y);
    }%

    \coordinate (O) at (0,0.1);
    \coordinate (x) at (\axislength,0.1);
    \coordinate (y) at (0,\axislength+.1);
    \draw[->] (O) -- (x) node[above] {$x$};
    \draw[->] (O) -- (y) node[above] {$y$};

    \fill[pattern=north west lines] (0,-.1) rectangle ++(6,.1);
    \fill[pattern=north west lines] (0,3) rectangle ++(6,.1);
  \end{tikzpicture}
  \caption{Couette flow between two parallel plates. The horizontal distance
  scale is $\gg L$, so the plates are considered infinite in the $x$ and $z$
  directions. The flow is here at a fully developed stage.}
  \label{fig:couette}
\end{figure}
The Navier-Stokes equations govern the time development of the fluid velocity field.
The momentum equation reads
\begin{align*}
  \pdv{\vb{u}}{t} + (\vb{u} \cdot \nabla) \vb{u} = - \frac{1}{\rho} \nabla p + \nu \nabla^2 \vb{u}
\end{align*}
where $\nu = \mu/\rho$ is the kinematic viscosity, and the incompressibility
condition is
\begin{align*}
  \nabla \cdot \vb{u} = 0
\end{align*}
The only force acting on the system is the tangential shear stress in the
horizontal direction, and the velocity field is therefore assumed to be on the
form $\vb{u} = (u,v = 0,w = 0)$. The pressure gradient term is eliminated because the
pressure is constant, and the inertial term vanishes because of the incompressibility
condition. The equations reduce to
\begin{align}
  \pdv{u}{t} = \nu \pdv[2]{u}{y}
\end{align}
which is the diffusion equation. The no-slip boundary condition applies to both
surfaces, so
\begin{align*}
  u(y = L) &= U \\
  u(y = 0) &= 0
\end{align*}
Other boundary conditions such as the no penetration condition (no velocity
normal to the boundaries at the boundaries) yield no additional information.
\section{Method} \label{sec:method}
\section{Results} \label{sec:results}
\section{Discussion} \label{sec:discussion}
\section{Conclusion} \label{sec:conclusion}
\bibliography{references}
\blankpage
\appendix
\section{Appendix section} \label{sec:label}

\blankpage
\end{document}

% Local Variables:
% TeX-engine: luatex
% End:
